% RELATED WORK

In this Chapter, we presented the relevant related work found in the literature for the topics addressed in the thesis. We offer a systematic analysis of current state-of-the art in cloud storage, accounting for the main driving forces and design issues behind it. Moreover, we complete this analysis with comparative tables and a final description on influential systems and their deployment. 

%% ARCHITECTURE

This Chapter described the core aspects of our HBase-QoD proposal, addressing its architecture, regarding system, network and software components. We also described the relevant aspects that make consistency enforcement more flexible and aware of user/developer semantics, driven by QoD consistency vectors, followed by the operation/update grouping semantics also provided.

%%% IMPLEMENTATION 

In this Chapter, we highlighted the most relevant implementation details, regarding the integration of the QoD consistency model as a module, into the inner workings of a fully operational HBase deployment. We also offer detail on the extensions of the more relevant inner HBase mechanisms, filtering, cache and consistency constrains upholding.
 

%%% Evaluation

In this chapter, we described the evaluation of HBase-QoD framework, regarding its performance, semantics, and resource utilization. This was carried out by comparative assessment between the original version of HBase (No-QoD) and HBase-QoD, making use of widely adopted benchmarks found in the literature.
The HBase-QoD prototype was evaluated with a test-bed of HBase clusters at INESC-ID and IST in Lisbon, some of them with an HBase-QoD enabled engine for quality of data between replicas, and others running a regular implementation of HBase 0.94.8.
Globally , the results were positive and in line with what was expected, across a variety of YCSB-derived workloads, regarding overall bandwidth utilization  and its peak usage, update latency (and application semantics), as well as CPU utilization.

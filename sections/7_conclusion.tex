%!TEX root = ../document.tex


\chapter{Conclusion}
\label{ch:conclusion}

\begin{quotation}
  "The last mile is always the most difficult, and (looking backwards) the best"
  {\small\it -- Miguel Mira Da Silva, professor at IST}
\end{quotation}

We end this article, making an overview and summing up all the primary aspects of the proposed work and how it relates to what has been researched so far, presenting also some concluding remarks.
People sharing resources is one of the oldest sociological behaviors in human history, however although some known attempts as SETI@HOME (even if extended with nuBOINC) have enabled that for our computer machinery, the level of friction that has to be made in order for a user to join, has been significantly high to cause a great user adoption. On the other hand, Open Cloud stacks have been evolving, providing nowadays the most reliable and distributed systems performance, having a bigger adoption even if the resources are geographically more distant or expensive.
The proposed work is an exercise to strive towards a federated community cloud that will enable its users to share effectively their resources, giving developers a reliable and efficient way to store and process data for their applications, with an API that is familiar to the centralized Cloud model.

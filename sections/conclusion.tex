  %%%%%%%%%%%%%%%%%%%%%%%%%%%%%%%%%%%%%%% -*- coding: utf-8; mode: latex -*- %%
  %
%%%%%                       CHAPTER
 %%%
  %

% $Id: 3200-adipisci-velit.tex,v 1.2 2007/11/23 10:14:56 david Exp $
% $Log: 3200-adipisci-velit.tex,v $
% Revision 1.2  2007/11/23 10:14:56  david
% Bug fixes galore.
%
% Revision 1.1  2007/11/23 09:52:43  david
% *** empty log message ***
%
%

  %%%%%%%%%%%%%%%%%%%%%%%%%%%%%%%%%%%%%%%%%%%%%%%%%%%%%%%%%%%%%%%%%%%%%%%%%%%%%
  %
%%%%%                    HEAD MATTER
 %%%
  %

\chapter{Conclusion}
%\addcontentsline{lof}{chapter}{\thechapter\quad Nihil Molestiae}
%\addcontentsline{lot}{chapter}{\thechapter\quad Nihil Molestiae}
\label{ch:conclusion}

\begin{quotation}
  "The last mile is always the most difficult, and (looking backwards) the best"
  {\small\it -- Miguel Mira Da Silva, professor at IST}
\end{quotation}



  %%%%%%%%%%%%%%%%%%%%%%%%%%%%%%%%%%%%%%%%%%%%%%%%%%%%%%%%%%%%%%%%%%%%%%%%%%%%%
  %
%%%%%                        FIRST SECTION.
 %%%
  %

\section{Concluding remarks}
To sum up the thesis briefly, Chapter 1 and 2 introduced the main ideas, topics and driving forces behind the thesis. Later, Chapter 3 and 4 dig into the main components of the proposed system and how they were designed, implemented and deployed. Finally, Chapter 5 offered evaluation measuring the resulting performance from the HBase-QoD paradigm introduced. This chapter closes the thesis with some conclusions regarding the work presented, and some lines of possible future work.

We started with an introductory chapter presenting the work domain of cloud tabular storage, the current shortcomings found in HBase, and the contribution proposed. Then, we have reviewed the most well-known and state of the art in replication for distributed systems, outlined the advantages and disadvantages of each of them.

Following that, we performed a deeper introspection into the mechanisms of the selected cloud data store in questions for this work, HBase, where we identify its weaknesses (including currently missing features) and introduce HBase-QoD in order to achieve bandwidth, and therefore cost, savings during replication, as shown in Chapter 5.

Finally, we believe in the re-usability of the solution developed, and the possibility to extend and adapt the framework to other cloud data stores, so a wider choice of consistency guarantees can be provided on top of our implementation, if further required by applications. This work is therefore useful and applicable, as a more flexible consistency model, to cloud data stores in cases where bandwidth is precious and cost savings mandatory.

Applied to the core of HBase for inter-datacenter scenarios, it provides users and applications with just the quality-of-data requested. On the other hand, administrators and developers can easily tune the bounds and the framework, in order to perform replication in a more fine-grained and timely fashion.

The same principle applies to cyclic multi-master scenarios, where each master acts as master and slave all at once. Although we did not test that or configure it in our slaves as we did find it critical, in order to provide a feasible proof of concept for the proposal.

To wrap up, have found this thesis to be a source of hard work and enthusiasm, as well as a valid motivation to interact more closely with the people from the HBase development community, with its material and results suitable to be submitted and accepted to international conferences, as well as a drive to engage in contacts and discussions with other research institutes.


  %%%%%%%%%%%%%%%%%%%%%%%%%%%%%%%%%%%%%%%%%%%%%%%%%%%%%%%%%%%%%%%%%%%%%%%%%%%%%

%%% Local Variables:
%%% mode: latex
%%% TeX-master: "tese"
%%% End:

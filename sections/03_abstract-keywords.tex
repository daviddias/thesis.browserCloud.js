%!TEX root = ../document.tex
\chapter*{Abstract}
\thispagestyle{empty}

Grid Computing has been around since the 90's, its fundamental basis is to use idle resources in geographically distributed systems in order to maximize their efficiency, giving researchers access to computational resources to perform their jobs (e.g. studies, simulations, rendering, data processing, etc). This approach quickly grew into non Grid environments, causing the appearance of projects such as SETI@Home or Folding@Home, leveraging volunteered shared resources and not only institution-wide data centers as before, giving the birth of Public Computing. Today, after having volunteering computing as a proven concept, we face the challenge of how to create a simple, effective, way for people to participate in such community efforts and even more importantly, how to reduce the friction of adoption by the developers and researchers to use and provide these resources for their applications. This thesis explores and proposes novel ways to enable end user machines to communicate, using recent Web technologies such as WebRTC, creating a simple API that is familiar to those used to develop applications for the Cloud, but with resources provided by a community and not by a company or institution.


\newpage
\chapter*{Resumo}
\thispagestyle{empty}

A "Grid Computing" está presente deste a década de 90, o seu objectivo fundamental é permitir a utilização de recursos inutilizados que se encontram distribuidos de forma geográficamente distribuida, permitindo optimizar a efficiencia de como são usados, permitindo que investigadores tenham acesso a estes recursos computacionais para executar tarefas (e.g. estudos, simulações, processamento de imagem, processamento de dados, etc). Este modelo cresceu rapidamente para ambientes não "Grid", dando origem a projectos como o "SETI@Home" ou o "Folding@Home", que tiram proveito de recursos partilhados voluntáriamente e não por apenas instituições, isto deu origem ao nascimento do que é conhecido como "Public Computing". Hoje, depois da partilha de recursos de forma voluntária ser um conceito provado, enfrentamos o desafio de criar uma forma simples e eficaz de mais pessoas poderem participar nestes esforços comunitários e ainda, com mais relevo, de reduzir o custo de entrada para a adopção por programadores e investigadores que querem usar e providenciar estes recursos para as aplicações. Esta tese explora e propõe novas formas de estabelecer um mecanismo de comunicação entre máquinas de utilizador, usando technologias recentes como é o caso do "WebRTC", criando uma "API" familiar para os programadores de aplicações da Computação na Nuvem, mas com recursos disponibilizados por uma comunidade e não por uma empresa ou instituição.

\newpage
\chapter*{}
\thispagestyle{empty}

\section*{Palavras Chave}
{\large
    \noindent Computa\c{c}\~ao na Nuvem,
    \noindent Redes entre pares,
    \noindent Computação voluntária,
    \noindent Partilha de ciclos,
    \noindent Computa\c{c}\~ao distribuida e descentralizada,
    \noindent Plataforma Web,
    \noindent Javascript
    \noindent Tolerância à faltas,
    \noindent Mecanismo de reputação,
    \noindent Nuvem comunitária
    \noindent WebRTC
}

\section*{Keywords}

{\large
    \noindent Cloud Computing,
    \noindent Peer-to-peer,
    \noindent Voluntary Computing,
    \noindent Cycle Sharing,
    \noindent Decentralized Distributed Systems,
    \noindent Web Platform,
    \noindent Javascript,
    \noindent Fault Tolerance,
    \noindent Reputation Mechanism,
    \noindent Community Cloud
    \noindent WebRTC
}


\cleardoublepage
\pagestyle{plain}
\pagenumbering{roman}
\def\contentsname{Index}
\tableofcontents
\newpage
\listoffigures
\newpage
\listoftables
\cleardoublepage

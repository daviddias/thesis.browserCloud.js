%!TEX root = ../document.tex
\chapter*{Abstract}
\thispagestyle{empty}

Grid computing has been around since the 90's, its fundamental basis is to use idle resources in geographically distributed systems in order to maximize its efficiency, giving researchers access to computational resources to perform their jobs (e.g. studies, simulations, rendering, data processing, etc). This approach quickly grew into non grid environments, causing the appearance of projects such as SETI@Home or Folding@Home, leveraging volunteered shared resources and not only institution-wide data centers as before, giving the birth of Public Computing. Today, after having volunteering computing as a proven concept, we face the challenge of how to create a simple, effective, way for people to participate in this community efforts and even more importantly, how to reduce the friction of adoption by the developers and researchers to use this resources for their application. This thesis explores and innovates new ways to enable end user machines to communicate, using recent Web technologies such as WebRTC, creating a simple API that is familiar to those used to develop applications for the Cloud, but with resources provided by a community and not by a company or institution.


\newpage
\chapter*{Resumo}
\thispagestyle{empty}

IGUAL AO ABSTRACT MAS EM PORTUGUÊS



\newpage
\chapter*{}
\thispagestyle{empty}

\section*{Palavras Chave}
{\large
    \noindent Computa\c{c}\~ao na Nuvem,
    \noindent Redes entre pares,
    \noindent Computação voluntária,
    \noindent Partilha de ciclos,
    \noindent Computa\c{c}\~ao distribuida e descentralizada,
    \noindent Plataforma Web,
    \noindent Javascript
    \noindent Tolerância à faltas,
    \noindent Mecanismo de reputação,
    \noindent Nuvem comunitária
    \noindent WebRTC
}

\section*{Keywords}

{\large
    \noindent Cloud Computing,
    \noindent Peer-to-peer,
    \noindent Voluntary Computing,
    \noindent Cycle Sharing,
    \noindent Decentralized Distributed Systems,
    \noindent Web Platform,
    \noindent Javascript,
    \noindent Fault Tolerance,
    \noindent Reputation Mechanism,
    \noindent Community Cloud
    \noindent WebRTC
}


\cleardoublepage
\pagestyle{plain}
\pagenumbering{roman}
\def\contentsname{Index}
\tableofcontents
\newpage
\listoffigures
\newpage
\listoftables
\cleardoublepage

\section{Future Work}
The work here concluded has covered several concepts and concerns in the area of Geo-Replication. Firstly, in the future, the evaluation could be extended, with a good experiment that would be to deploy and execute HBase-QoD on Amazon EC2, with various setups, as well as using EC2 for larger stress testing and benchmarking.

Following up, and in terms of cost savings and performance, it would be interesting to apply these same concepts and ideas, and dig deeper into innovative and rising areas of Big Data research such as Green Computing. This, naturally including working metrics based on relevant environmental aspects, such as the impact of CPU intensive replication tasks have into carbon footprint and power-efficiency for large geographically distributed cloud data centers.

A great addition to this thesis would also be the development of a performance model and data analysis framework for different cloud scenarios, by using HBase-QoD in a next generation, in order to support measurement consistency in relation to response times, fairness and power-consumption. 

Besides that, elasticity is nowadays a key metric on cloud deployments, therefore, introducing the concept of auto-scaling for sets of replicas would be also advantageous in a further effort to evaluate a trade-off between replication and server side CPU cycles.     
%!TEX root = ../document.tex
\chapter{Introduction}\label{ch:intro}

\begin{quotation}
    "OUT OF THIS WORLD QUOTE".
    {\small\it 
        -- ~\footnote{Lehene C. HStack, http://hstack.org/why-were-using-hbase-part-2} }
\end{quotation}

\section{Overview}

Today, in the information communications technology landscape, with the introduction of social networks, search engines, Internet of things, which led to innovation on home and vehicle automation, user data has been growing at a large pace. The store, transfer, processing and analysis of all this data brings considerable new knowledge breakthroughs, enabling us to optimize systems towards a better and enhanced experience. However, how to use the information available to achieve these breakthroughs has been one of the main challenges since then. 

\section{Problem Statement}

The resources required to execute a contiguous and significant analysis of the data available are controlled by a small subset of comapnies and organizations. In order to enable more people to use Big Data analysis, we need to reduce the cost that is inherent to process all this user information, which typically need big amounts of CPU cycles for processing, analysis and inference.

Currently addressing this issues in part, Cloud Computing has revolutionized the computing landscape due to key advantages to developers/users over pre-existing computing paradigms, the main reasons are:

\begin{itemize}
    \item Virtually unlimited scalability of resources, avoiding disruptive infrastructure replacements.
    \item Utility-inspired pay-as-you-go and self-service purchasing model, minimizing capital expenditure.
    \item Virtualization-enabled seamless usage and easier programming interfaces.
    \item Simple, portable internet service based interfaces, straightforward for non expert users, enabling adoption and use of cloud services without any prior training.
\end{itemize}

Grid computing had offered before a solution for high CPU bound computations, however it has high entry barriers, being necessary to have a large infrastructure, even if just to execute small or medium size computing jobs. Cloud computing solves this by offering a solution ``pay-as-you-go'', which transformed computing into an utility.

Still, even though we are able to integrate several Cloud providers into an open software stack, Cloud computing relies nowadays on centralized architectures, resorting to data centers, using mainly the Client-Server model. In this work, we pursue a shift in this paradigm.

Unlike the conventional approach to make Cloud Computing `green' (i.e. Green Computing) by improving datacenter's efficiency through expensive and strictly centralized control, our vision entails a shift in perspective, by enabling each user to contribute to this effort, leveraging his/her idle computing resources (sometimes up to 70\% of power wasted), and thus reducing overall environmental footprint. Thus browserCloud.js resources are provided in a voluntary manner by common Internet users that want to share their idle computer cycles and storage available, while browsing the web, without having the concern to setup any application or system to do so.

Community Clouds are not a novelty in the Distributed Systems research area. However, existing models have been developed to follow the client-server model, transporting the data to the place where the computation will take place, which causes big bottlenecks in network traffic, limiting the amount of computed units done in a delimited window of time. One of browserCloud.js goals is exactly to mitigate this bottleneck by taking the computation (the algorithms that will perform operations over the data) to the machines where the data is stored.

\section{Extended motivation and Roadmap}

Our main goal with this work is to design and implement a system that is able to take advantage of volunteered computer cycles through the most ubiquitous growing platform, the browser. In order to create this system, several components will be developed:

\begin{itemize}
    \item An efficient local storage module that offers persistence and availability, using browser storage for fast indexing.
    \item A distributed job scheduler able to receive jobs and coordinate with the nodes inside the network, without having to recur to a centralized control system.
    \item A job executioner able to receive different assets to perform the jobs (image/video manipulation, calculation, etc), taking advantage of the dynamic runtime available by the predominant language in the browser, javascript.
    \item A server to work as the entry point for browser to download the code necessary to run browserCloud.js logic. This is the only point that is considered to be centralized in the network, due to the limitation of browsers being typically behind NAT and not having static IPs
    \item Structured peer-to-peer overlay network for browsers to communicate directly among themselves, without being necessary to take the data or the computation to a centralized system.
    \item A client API, RESTful, so it is easy to develop applications for Desktops and mobile platforms without having to change the codebase or building a new SDK
    \item A command line interface for access like `mountable' partition to the storage in browserCloud.js, able to dispatch jobs in a very Unix way, by piping the results from one task to another task.
\end{itemize}

These components are fully described in section 4. After its development, a proposed evaluation is going to be executed, according to a set of assessment metrics, enabling us to compare the viability of browserCloud.js as a Cloud provider, comparing to existing centralized Cloud systems.

\section{Research Proposal}

To accomplish this, we propose a new approach to abandon the classic centralized Cloud Computing paradigm, towards a common, dynamic, and privacy-aware cloud infrastructure. This, by means of a fully decentralized architecture, federating freely ad-hoc distributed and heterogeneous resources, with instant effective resource usage and progress. Additional goals may include: arbitration, service-level agreements, resource handover, compatibility and maximization of host's and user's criteria, and cost- and carbon-efficiency models.

This work will address extending the Web Platform with technologies such as: WebRTC, Emscripten, Javascript and IndexedDB to create a structured peer-to-peer overlay network, federating ad-hoc personal resources into a geo-distributed cloud infrastructure, representing the definition made by C.Shirky of what an peer-to-peer means:

  \textit{``An application is peer-to-peer if it aggregates resources at the network’s edge, and those resources can be anything. It can be content, it can be cycles, it can be storage space, it can be human presence.''}, C.Shirky \cite{Shirky.}

Finally, browserCloud.js has the possibility to grow organically with the number of users. The management of these resources is done by an RESTful API, enabling desktop and mobile apps to use the resources available in a way that's familiar to developers. 

\section{Contributions}

List of open source modules
tecnology inspired by

\section{Publications}

Talks Delivered 

\section{Structure of the thesis}

\textbf{Document roadmap:} We start by describing the objectives of our solution in Section 2, and then, in Section 3 we present the state of the art for the technologies and areas of study relevant for he proposed work, which are: Cloud computing and Open Source Cloud Platforms (at 3.1), Volunteered resource sharing (at 3.2) and Resource sharing using the Web platform (at 3.3). In Section 4, we present the proposed architecture and respective software stack, moving to the system evaluation present on Section 5.



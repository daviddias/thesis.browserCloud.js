%%%%%%%%%%%%%%%%%%%%%%%%%%%%%%%%%%%%%%%%%%%%%%%%%%%%%%%%%%%%%%%%%%%%%%%%%%%%%
%%%%%                  THESIS EMDC ALVARO GARCIA RECUERO
%%%
%

\def\date{September 2013}
\def\titulo{HBase-QoD: Vector-Field Consistency for Replicated Cloud Storage}

% hypernavigation in PDF docs
\hypersetup{colorlinks,
   debug=false,
   linkcolor=blue,  %%% cor do tableofcontents, \ref, \footnote, etc
   citecolor=red,  %%% cor do \cite
   urlcolor=blue,   %%% cor do \url e \href
   bookmarksopen=true,
   pdftitle={\titulo},
   pdfauthor={Álvaro García Recuero},
   pdfsubject={European Master in Distributed Systems},
   pdfkeywords={Cloud Computing, Replication, Consistency, Quality of Data}
}

  %%%%%%%%%%%%%%%%%%%%%%%%%%%%%%%%%%%%%%%%%%%%%%%%%%%%%%%%%%%%%%%%%%%%%%%%%%%%%
  %
%%%%%                          CAPA DA TESE
 %%%
  %

\thispagestyle{empty}

\begin{singlespace}
\vbox to\textheight{%
%--------------------------------------------------
\vskip-1.3in%---------- LOGO AND NAME IST/UTL -------
%--------------------------------------------------
\hskip-17mm\vbox to50mm{
\vfil
\begin{tabular}{l}
\includegraphics[width=5cm]{figs/preliminar/Logo_IST_web.eps}
\end{tabular}
\vfil
\vfil
}%
%--------------------------------------------------
\vskip18mm%---------- FIGURES DA CAPA -------------
%--------------------------------------------------
\vbox to25mm{\LARGE\sl
\vfil
%\centerline{\psfig{file=figs/preliminar/tarantula.eps,height=25mm}}
\vfil
}%
%--------------------------------------------------
\vskip6mm%---------- TITLE -----------------------
%--------------------------------------------------
\vbox to25mm{\LARGE\bf
\vfil
\begin{center}
\titulo
\end{center}
\vfil
}%
%--------------------------------------------------
\vskip10mm%---- NAME AND ACTUAL GRADE  -----------
%--------------------------------------------------
\vbox to25mm{\large
\vfil
\begin{center}
{\Large\bf Álvaro García Recuero}\\   % author's name
\end{center}
\vfil
}%
%--------------------------------------------------
\vskip8mm%---------- GRADE TO OBTAIN  -------------
%--------------------------------------------------
\vbox to8mm{\large
\vfil
\centerline{Thesis to obtain the Master of Science Degree in}
%\vskip6mm
{\LARGE\bf \centerline{Information Systems and Computer Engineering}}
\vfil
}%
%--------------------------------------------------
\vskip10mm%---------- ORIENTADOR -------------------
%--------------------------------------------------
%\vbox to8mm{\large
%\vfil
%\begin{center}
%\begin{tabular}{p{0.2\textwidth}l}
%\end{tabular}
%\end{center}
%\vfil
%}%
%%--------------------------------------------------
%\vfil
% %--------------------------------------------------
% \vskip5mm%---------- JÚRI -------------------------
% %--------------------------------------------------
\vbox to7mm{\Large\bf
\vfil
\begin{center}
{\Large\bf Examination Committee}\\
\end{center}
\vfil
}%

\vbox to28mm{\large
\vfil
\begin{center}
%\begin{tabular}{p{0.2\textwidth}l}
%Chairman: & Lu\'{i}s Eduardo Teixeira Rodrigues\\
%Supervisor: & Lu\'{i}s Manuel Antunes Veiga\\
%Member: & Johan Montelius\\
%\end{tabular}
Chairperson: Prof. Doutor. Lu\'{i}s Eduardo Teixeira Rodrigues\\
Supervisor: Prof. Doutor. Lu\'{i}s Manuel Antunes Veiga\\
Member of the Committee: Prof. Doutor. Johan Montelius\\
\end{center}
\vfil
}%
%--------------------------------------------------
\vskip28mm%---------- DATA -------------------------
%--------------------------------------------------
\vbox to4mm{\Large\bf
\vfil
\begin{center}
\fontsize{14pt}{12pt}\selectfont \date
\end{center}
\vfil
}%
%--------------------------------------------------
}%vbox
\end{singlespace}
\newpage

  %%%%%%%%%%%%%%%%%%%%%%%%%%%%%%%%%%%%%%%%%%%%%%%%%%%%%%%%%%%%%%%%%%%%%%%%%%%%%
  %
%%%%%                             AGRADECIMENTOS
 %%%
  %


\chapter*{Acknowledgements}
%\chapter*{Acknowledgements}
\thispagestyle{empty}

% AGRADECER!
%\vspace{-40pt}
The work here presented is delivered as final thesis report at Instituto Superior Técnico (IST) in Lisbon, Portugal and it is in partial fulfillment of the European Master in Distributed Computing belonging to promotion of 2011-2013. The Master programme has been composed of a first year at IST, a second year's first semester at Royal Institute of Technology (KTH) and for this work and last academic term, based at the research lab INESC-ID Lisbon with the support of a the scientific grant PTDC/EIA-EIA/108963/2008 as part of a national project (RepComp) funded by FCT Portugal.

Special thanks to my thesis supervisor and coordinator from the European Master of Distributed Computing Luís Veiga. I am also grateful to Sergio Esteves for sharing advice and brainstorming sessions with me during this period at INESC-ID.

I am thankful for having a family which always supports my decision. Thank you for always been been there and supporting me in the most difficult moments during the last two years. I also feel lucky for having a brother who has the ability to always making me get a laugh and relax so I keep going in despite of the most stressing or demanding situation.

Last but not least, to all the professors from IST and KTH that during these last two years challenged me to think out of the box and face always the difficulties in despite of any other matters, they taught me to always work towards bigger and better goals. Thank you to Luís Eduardo Teixeira Rodrigues, Paulo Jorge Pires Ferreira, João Coelho Garcia, Carlos Nuno da Cruz Ribeiro and Johan Montelius respectively.

As a key mention, I would also like to be thankful here to Lars Hofhansl (larsh@apache.org) from the Apache Foundation, particularly helpful in guiding me in the early stages of this research for directions through the HBase code base and so, I really appreciate his supportive and unselfishness attitude sharing expert advice with me.

\vspace{15pt}
\vfill
\begin{flushright}
  \begin{minipage}{8cm}
    \begin{center}
    20th of September, Lisbon \\ 
	Álvaro García Recuero
    \end{center}
  \end{minipage}
\end{flushright}

\cleardoublepage

  %%%%%%%%%%%%%%%%%%%%%%%%%%%%%%%%%%%%%%%%%%%%%%%%%%%%%%%%%%%%%%%%%%%%%%%%%%%%%
  %
%%%%%                            DEDICATÓRIAS
 %%%
  %

\chapter*{}
\thispagestyle{empty}

% DEDICAR!
\vfill
\mbox{}
\vfill\Large
\begin{flushright}
  \begin{minipage}{8cm}
    \begin{center}

--To my family
%"The last mile is always the most difficult, and (looking backwards) the best" -- Miguel Mira Da Silva (professor at IST)

    \end{center}
  \end{minipage}
\end{flushright}
\normalsize\vfill

\cleardoublepage

  %%%%%%%%%%%%%%%%%%%%%%%%%%%%%%%%%%%%%%%%%%%%%%%%%%%%%%%%%%%%%%%%%%%%%%%%%%%%%
  %
%%%%%                                RESUMO
 %%%
  %


  %%%%%%%%%%%%%%%%%%%%%%%%%%%%%%%%%%%%%%%%%%%%%%%%%%%%%%%%%%%%%%%%%%%%%%%%%%%%%
  %
%%%%%                            ABSTRACT
 %%%
  %

\chapter*{Abstract}
\thispagestyle{empty}

Many of today's applications deployed in cloud computing environments make use of key-value storage such as BigTable, Cassandra, and many other no-SQL approaches to overcome scalability limits of relational databases. Relevant open-source solutions include Apache HBase. Several works such as Percolator notify applications whenever data is updated by others (e.g., in the context of updating Google's web index).

For increased performance and scalability, such storage is partitioned across machines and data centers, and each node's data is replicated for availability therefore. Furthermore, fragments of the key-value store should be geo-cached as close as possible to the edge of the network location for increased performance and to reduce the load on mega data centers.

This work aims at extending HBase with client-centric caching and replication policies in regards to a consistency model based on data divergence bounds and user-defined application semantics, which we define as Quality-of-Data (QoD). Thus, data stored at HBase-QoD will be kept in the master of a data center with possibly several cached replicas on the slaves region servers.

Overall, the data may have different consistency guarantees and synchronization requirements that will be applicable to inter-replication with other master servers or clusters. This reduces the number of messages and bandwidth needed by master servers to notify applications of data changes and replica updates, while still being able to fulfill those data-defined semantics according to a vector-field consistency named HBase-QoD.

%Using this paradigm we can realize bandwidth savings in simulated peak loads scenarios in regard to high overhead of replication between data centers. We still provide different levels of consistency to different users or applications needs (e.g., regarding timeliness, number of pending updates and divergence bounds). We verify the implementation of the Quality-of-Data based vector on a set of distributed clusters. Levels of data consistency are enforced while performance of the cluster is monitored to ensure it remains stable, and updates can be tagged and grouped atomically in logical batches, akin to transactions.
%
%
%
%There has been extensive recent research in distributed systems regarding consistency in geo-replicated systems. NoSQL databases have introduced a new paradigm in this field, where user data location and replication mechanisms work together to be able to offer best-in-class distributed applications. As such, that means running costly operations under lower system latency while still ensuring data read is still valid for application logic processing. Therefore, several consistency models (like the eventual) use a flexible data consistency approach, that consists in taking advantage of those where staleness of data can be tolerated temporarily.
%
%We materialize that the idea in the context of an extensible record store. We assign priority to data depending of data semantics and serve it to applications when it is just required, avoiding overloading the network during large periods of disconnection or partitions. It is a practical and flexible approach that currently is not implemented in HBase as such, but even though, it is a desired feature in some community forums of the Apache project itself.
%
%We introduce the data semantics used for non-relational data stores and built around the core architecture of an existing well-known system as such, enabling it to trigger data replication selectively with the support of a three-dimensional field vector. In addition, grouping of operations is possible, ensuring atomic changes and therefore valid updates are propagated.
%
%Using this paradigm we can realize bandwidth savings in simulated peak loads scenarios in regard to high overhead of replication between data centers. We still provide different levels of consistency to different users or applications needs (e.g., regarding timeliness, number of pending updates and divergence bounds). We verify the implementation of the Quality-of-Data based vector on a set of distributed clusters. Levels of data consistency are enforced while performance of the cluster is monitored to ensure it remains stable, and updates can be tagged and grouped atomically in logical batches, akin to transactions.

\newpage
\chapter*{Resumo}
\thispagestyle{empty}

Muitas das aplicações actualmente disponibilizadas em ambientes de computação em nuvem fazem uso de sistemas de armazenamento associativo chave-valor, tais como o BigTable, Cassandra, e muitos outros baseados em abordagens no-SQL para contornar as limitações de escalabilidade das bases de dados relacionais.

%Implementações open-source relevantes incluem o Apache HBase. Alguns projectos como o Percolator notificam as aplicações sempre que os dados são actualizados por outras aplicações (por ex. no contexto do cálulo do web index Google).

Para melhorar o desempenho e a escalabilidade, os sistemas de armazenamento são particionados por vários servidores, e centros de dados, com os dados de cada servidor replicados para assegurar disponibilidade. Além disso, parcelas do repositório chave-valor devem ser mantidas geo-cached tão perto quanto possível da periferia da rede, para maior desempenho e para reduzir a carga nos mega centros de dados.

Este trabalho tem como objectivo estender o HBase com políticas de caching e de replicação centradas no cliente, com um modelo de consistência baseado em limitação da divergência dos dados e na semântica das aplicações, que definimos como Quality-of-Data (QoD). Assim, os dados armazenados no HBase-QoD serão mantidos na réplica principal de um centro de dados com possivelmente várias replicas  secundárias denominadas region servers.

Globalmente, os dados podem obedecer a diferentes garantias de consistência e requisitos de sincronização, que serão aplicados na replicação entre centros de dados. Isto reduz o número de mensagens e largura de banda necessárias às réplicas para notificar aplicações de modificações nos dados ou actualizações. Isto, enquanto sendo capaz de fazer cumprir a semânctica definida pelas aplicações de acordo com um modelo vectorial de consistência denominado HBase-QoD.

\newpage
  %%%%%%%%%%%%%%%%%%%%%%%%%%%%%%%%%%%%%%%%%%%%%%%%%%%%%%%%%%%%%%%%%%%%%%%%%%%%%
  %
%%%%%                 FICHA BIBLIOGRAFICA -- PALAVRAS CHAVE
 %%%
  %

\chapter*{}
\thispagestyle{empty}

\section*{Palavras Chave}
{\large % EM PORTUGUÊS

\noindent Geo-Replica\c{c}\~ao

\noindent Bases de Dados NoSQL

\noindent HBase

\noindent Consist\^encia Adapt\'avel %Tunable Consistency

\noindent Qualidade de Dados

\noindent Divergência de Réplicas em Sistemas Geo-replicados
}

\section*{Keywords}

{\large % EM INGLÊS

\noindent Geo-Replication

\noindent NoSQL Databases

\noindent HBase

\noindent Eventual Consistency

\noindent Quality of Data

\noindent Tunable Consistency

\noindent Divergence-bounding

}

\vfill
%LATEX2HTML}

\cleardoublepage


  %%%%%%%%%%%%%%%%%%%%%%%%%%%%%%%%%%%%%%%%%%%%%%%%%%%%%%%%%%%%%%%%%%%%%%%%%%%%%
  %
%%%%%                         MUDANÇA DE NUMERAÇÃO
 %%%
  %

\pagestyle{plain}
\pagenumbering{roman}

  %%%%%%%%%%%%%%%%%%%%%%%%%%%%%%%%%%%%%%%%%%%%%%%%%%%%%%%%%%%%%%%%%%%%%%%%%%%%%
  %
%%%%%                             INDICES
 %%%
  %

% ``Table of contents'' (índice).

\def\contentsname{Index}
\tableofcontents
\newpage

% Lista de figuras.
\listoffigures
\newpage

% Lista de tabelas.
\listoftables

% Does it always work? I expect so...
\cleardoublepage

  %
 %%%
%%%%%                          F          I          M
  %
  %%%%%%%%%%%%%%%%%%%%%%%%%%%%%%%%%%%%%%%%%%%%%%%%%%%%%%%%%%%%%%%%%%%%%%%%%%%%%

% Local Variables:
% mode: latex
% TeX-master: "tese"
% End:

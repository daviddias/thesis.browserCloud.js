%!TEX root = ../document.tex


\def\date{March 2015}
\def\titulo{browserCloud.js - A federated community cloud served by a P2P overlay network on top of the web platform}

\hypersetup{colorlinks,
   debug=false,
   linkcolor=blue,
   citecolor=red,
   urlcolor=blue,
   bookmarksopen=true,
   pdftitle={\titulo},
   pdfauthor={David Dias}
   pdfsubject={Master of Science in P2P Networks, Cloud computing and Mobile Applications},
   pdfkeywords={Cloud Computing, Peer-to-peer, Voluntary Computing, Cycle Sharing, Decentralized Distributed Systems, Web Platform, Javascript, Fault Tolerance, Reputation Mechanism, Community Cloud}
}

\thispagestyle{empty}
\begin{singlespace}
\vbox to\textheight{%
\vskip-1.3in
\hskip-17mm\vbox to50mm{
\vfil
\begin{tabular}{l}
\includegraphics[width=5cm]{figs/preliminar/Logo_IST_web.eps}
\end{tabular}
\vfil
\vfil
}%
\vskip18mm%---------- COVER IMAGE -------------
\vbox to25mm{\LARGE\sl
\vfil
%\centerline{\psfig{file=figs/preliminar/tarantula.eps,height=25mm}}
\vfil
}%
\vskip6mm%---------- TITLE -----------------------
\vbox to25mm{\LARGE\bf
\vfil
\begin{center}
\titulo
\end{center}
\vfil
}%
\vskip10mm%---- NAME AND ACTUAL GRADE  -----------
\vbox to25mm{\large
\vfil
\begin{center}
{\Large\bf David Dias}\\
\end{center}
\vfil
}
\vskip8mm%---------- GRADE TO OBTAIN  -------------
\vbox to8mm{\large
\vfil
\centerline{Thesis to obtain the Master of Science Degree in P2P Networks, Cloud computing and Mobile Applications}
%\vskip6mm
{\LARGE\bf \centerline{BSc in Communication Networks}}
\vfil
}
\vskip10mm%---------- JURI -----------
\vbox to7mm{\Large\bf
\vfil
\begin{center}
{\Large\bf Examination Committee}\\
\end{center}
\vfil
}

\vbox to28mm{\large
\vfil
\begin{center}
Chairperson: Prof. Doutor. \\
Supervisor: Prof. Doutor. Lu\'{i}s Manuel Antunes Veiga\\
Member of the Committee: Prof. Doutor. João Dias Pereira\\
\end{center}
\vfil
}
\vskip28mm%---------- DATE -------------------------
\vbox to4mm{\Large\bf
\vfil
\begin{center}
\fontsize{14pt}{12pt}\selectfont \date
\end{center}
\vfil
}
}
\end{singlespace}
\newpage




\chapter*{Acknowledgements}
\thispagestyle{empty}

WHO YOU ARE THANKFUL TO :D

\vspace{15pt}
\vfill
\begin{flushright}
  \begin{minipage}{8cm}
    \begin{center}
    20th of September, Lisbon \\ 
    David Dias
    \end{center}
  \end{minipage}
\end{flushright}

\cleardoublepage

\chapter*{}
\thispagestyle{empty}
\vfill
\mbox{}
\vfill\Large
\begin{flushright}
  \begin{minipage}{8cm}
    \begin{center}

--To all of the first followers, you undoubtly changed my life.

    \end{center}
  \end{minipage}
\end{flushright}
\normalsize\vfill

\cleardoublepage







\chapter*{Abstract}
\thispagestyle{empty}

Grid computing has been around since the 90's, its fundamental basis is to use idle resources in geographically distributed systems in order to maximize its efficiency, giving researchers access to computational resources to perform their jobs (e.g. studies, simulations, rendering, data processing, etc). This approach quickly grew into non grid environments, causing the appearance of projects such as SETI@Home or Folding@Home, that use volunteered shared resources and not only institution-wide data centers as before, creating the concept of Public Computing. Today, after having volunteering computing as a proven concept, we face the challenge of how to create a simple, effective, way for people to participate in this community efforts and even more importantly, how to reduce the friction of adoption by the developers and researchers to use this resources for their applications. This work explores current ways of making an interopable way of end user machines to communicate, using new Web technologies, creating a simple API that is familiar to those used to develop applications for the Cloud, but with resources provided by a community and not by a company or institution.


\newpage
\chapter*{Resumo}
\thispagestyle{empty}

IGUAL AO ABSTRACT MAS EM PORTUGUÊS
IGUAL AO ABSTRACT MAS EM PORTUGUÊS
IGUAL AO ABSTRACT MAS EM PORTUGUÊS
IGUAL AO ABSTRACT MAS EM PORTUGUÊS
IGUAL AO ABSTRACT MAS EM PORTUGUÊS
IGUAL AO ABSTRACT MAS EM PORTUGUÊS
IGUAL AO ABSTRACT MAS EM PORTUGUÊS
IGUAL AO ABSTRACT MAS EM PORTUGUÊS
IGUAL AO ABSTRACT MAS EM PORTUGUÊS
IGUAL AO ABSTRACT MAS EM PORTUGUÊS


\newpage

\chapter*{}
\thispagestyle{empty}

\section*{Palavras Chave}
{\large % EM PORTUGUÊS

\noindent Computa\c{c}\~ao na Nuvem,
\noindent Redes entre pares,
\noindent Computação voluntária,
\noindent Partilha de ciclos,
\noindent Computa\c{c}\~ao distribuida e descentralizada,
\noindent Plataforma Web,
\noindent Tolerância à faltas,
\noindent Mecanismo de reputação,
\noindent Nuvem comunitária
}

\section*{Keywords}

{\large % EM INGLÊS

\noindent Cloud Computing,
\noindent Peer-to-peer,
\noindent Voluntary Computing,
\noindent Cycle Sharing,
\noindent Decentralized Distributed Systems,
\noindent Web Platform,
\noindent Javascript,
\noindent Fault Tolerance,
\noindent Reputation Mechanism,
\noindent Community Cloud
}

\vfill

\cleardoublepage
\pagestyle{plain}
\pagenumbering{roman}
\def\contentsname{Index}
\tableofcontents
\newpage
\listoffigures
\newpage
\listoftables
\cleardoublepage
